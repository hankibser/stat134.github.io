\documentclass[11pt]{amsart}

\usepackage{amsmath}
\usepackage{amsthm}
\usepackage{amsfonts}
\usepackage{amssymb}
\usepackage{enumerate}

\usepackage{graphics}
\usepackage{epsfig}
\usepackage{color}
\usepackage{verbatim}
%\parindent = 5 pt
%\parskip = 12 pt
%\oddsidemargin 0.1in \evensidemargin 0.1in \textwidth=7in
%\textheight=8.5in \itemsep=0in
%\parsep=0.1in

\textwidth 16cm \textheight 24cm \oddsidemargin 0.1cm
\evensidemargin 0.1cm \topmargin -0.8cm
%\def\yy{{\hbox{\bf y}}}
%\newcommand\eps{\varepsilon}
%\newcommand\tr{\operatorname{tr}}
\newcommand\diag{\operatorname{diag}}\newcommand\hess{\operatorname{Hess}}
\newcommand\divergence{\operatorname{div}}
\newcommand\ad{\operatorname{ad}}
%\newcommand\dist{\operatorname{dist}}
%\renewcommand\span{\operatorname{span}}
\newcommand\rank{\operatorname{rank}}
%\renewcommand\Re{\operatorname{Re}}
%\renewcommand\Im{\operatorname{Im}}
\newcommand\g{{\mathbb{g}}}
\newcommand\R{{\mathbb{R}}}
\newcommand\C{{\mathbb{C}}}
\newcommand\Z{{\mathbf{Z}}}
\newcommand\D{{\mathbf{D}}}
\newcommand\Q{{\mathbf{Q}}}
\newcommand\G{{\mathbf{G}}}
\newcommand\I{{\mathbf{I}}}
\renewcommand\P{{\mathbf{P}}}
\newcommand\E{{\mathbf{E}}}
\newcommand\U{{\mathbf{U}}}
\newcommand\A{{\mathbf{A}}}
\newcommand\Var{\mathbf{Var}}
\renewcommand\Im{{\operatorname{Im}}}
\renewcommand\Re{{\operatorname{Re}}}
\newcommand\eps{{\varepsilon}}
\newcommand\trace{\operatorname{trace}}
\newcommand\supp{\operatorname{supp}}
\newcommand\tr{\operatorname{tr}}
\newcommand\dist{\operatorname{dist}}
\newcommand\Span{\operatorname{Span}}
\newcommand\sgn{\operatorname{sgn}}
\newcommand\dive{\operatorname{div}}
\renewcommand\a{x}
\renewcommand\b{y}

% \newcommand\bm{{\mathbf{m}}}
\newcommand\dd{\partial}
\newcommand\smd{\hbox{smd}}
\newcommand\Ba{{\mathbf a}}
\newcommand\Bb{{\mathbf b}}
\newcommand\Bc{{\mathbf c}}
\newcommand\Bd{{\mathbf d}}
\newcommand\Be{{\mathbf e}}
\newcommand\Bf{{\mathbf f}}
\newcommand\Bg{{\mathbf g}}
\newcommand\Bh{{\mathbf h}}
\newcommand\Bi{{\mathbf i}}
\newcommand\Bj{{\mathbf j}}
\newcommand\Bk{{\mathbf k}}
\newcommand\Bl{{\mathbf l}}
\newcommand\Bm{{\mathbf m}}
\newcommand\Bn{{\mathbf n}}
\newcommand\Bo{{\mathbf o}}
\newcommand\Bp{{\mathbf p}}
\newcommand\Bq{{\mathbf q}}
\newcommand\Bs{{\mathbf s}}
\newcommand\Bt{{\mathbf t}}
\newcommand\Bu{{\mathbf u}}
\newcommand\Bv{{\mathbf v}}
\newcommand\Bw{{\mathbf w}}
\newcommand\Bx{{\mathbf x}}
\newcommand\By{{\mathbf y}}
\newcommand\Bz{{\mathbf z}}
\newcommand\p{{\mathbf p}}
\newcommand\bi{{\mathbf i}}
%

\newcommand\BA{{\mathbf A}}
\newcommand\BB{{\mathbf B}}
\newcommand\BC{{\mathbf C}}
\newcommand\BD{{\mathbf D}}
\newcommand\BE{{\mathbf E}}
\newcommand\BF{{\mathbf F}}
\newcommand\BG{{\mathbf G}}
\newcommand\BH{{\mathbf H}}
\newcommand\BI{{\mathbf I}}
\newcommand\BJ{{\mathbf J}}
\newcommand\BK{{\mathbf K}}
\newcommand\BL{{\mathbf L}}
\newcommand\BM{{\mathbf M}}
\newcommand\BN{{\mathbf N}}
\newcommand\BO{{\mathbf O}}
\newcommand\BP{{\mathbf P}}
\newcommand\BQ{{\mathbf Q}}
\newcommand\BS{{\mathbf S}}
\newcommand\BT{{\mathbf T}}
\newcommand\BU{{\mathbf U}}
\newcommand\BV{{\mathbf V}}
\newcommand\BW{{\mathbf W}}
\newcommand\BX{{\mathbf X}}
\newcommand\BY{{\mathbf Y}}
\newcommand\BZ{{\mathbf Z}}
\renewcommand\Pr{{\mathbf P }}

%cal letter

\newcommand\CA{{\mathcal A}}
\newcommand\CB{{\mathcal B}}
\newcommand\CC{{\mathcal C}}
\newcommand\CD{{\mathcal D}}
\newcommand\CE{{\mathcal E}}
\newcommand\CF{{\mathcal F}}
\newcommand\CG{{\mathcal G}}
\newcommand\CH{{\mathcal H}}
\newcommand\CI{{\mathcal I}}
\newcommand\CJ{{\mathcal J}}
\newcommand\CK{{\mathcal K}}
\newcommand\CL{{\mathcal L}}
\newcommand\CM{{\mathcal M}}
\newcommand\CN{{\mathcal N}}
\newcommand\CO{{\mathcal O}}
\newcommand\CP{{\mathcal P}}
\newcommand\CQ{{\mathcal Q}}
\newcommand\CS{{\mathcal S}}
\newcommand\CT{{\mathcal T}}
\newcommand\CU{{\mathcal U}}
\newcommand\CV{{\mathcal V}}
\newcommand\CW{{\mathcal W}}
\newcommand\CX{{\mathcal X}}
\newcommand\CY{{\mathcal Y}}
\newcommand\CZ{{\mathcal Z}}

%number theory

\newcommand\condo{{\bf C0}}
\newcommand\condone{{\bf C1}}

\newcommand\N{{\mathbb N}}
\newcommand\BBQ {{\mathbb Q}}
\newcommand\BBI {{\mathbb I}}
\newcommand\BBR {{\mathbb R}}
\newcommand\BBZ {{\mathbb Z}}
\newcommand\BBC{{\mathbb C}}
% tilde

\newcommand\ta{{\tilde a}}
\newcommand\tb{{\tilde b}}
\newcommand\tc{{\tilde c}}
\newcommand\td{{\tilde d}}
\newcommand\te{{\tilde e}}
\newcommand\tf{{\tilde f}}
\newcommand\tg{{\tilde g}}
% \newcommand\th{{\tilde h}}
\newcommand\ti{{\tilde i}}
\newcommand\tj{{\tilde j}}
\newcommand\tk{{\tilde k}}
\newcommand\tl{{\tilde l}}
\newcommand\tm{{\tilde m}}
\newcommand\tn{{\tilde n}}
%\newcommand\to{{\tilde o}}
\newcommand\tp{{\tilde p}}
\newcommand\tq{{\tilde q}}
\newcommand\ts{{\tilde s}}
%\newcommand\tt{{\tilde t}}
\newcommand\tu{{\tilde u}}
\newcommand\tv{{\tilde v}}
\newcommand\tw{{\tilde w}}
\newcommand\tx{{\tilde x}}
\newcommand\ty{{\tilde y}}
\newcommand\tz{{\tilde z}}
\newcommand\ep{{\epsilon}}
%\newcommand\trace{{\operatorname{trace}}}
\newcommand\sign{{\operatorname{sign}}}
%\newcommand\hess{{\operatorname{Hess}}}
\newcommand\Dyson{{\operatorname{Dyson}}}
\renewcommand\th{{\operatorname{th}}}






\def\x{{\bf X}}
\def\y{{\bf Y}}
%\def\I#1{{\bf 1}_{#1}}
\def\mb{\mbox}

% \swapnumbers
% \pagestyle{headings}
\parindent = 5 pt
\parskip = 12 pt

\theoremstyle{plain}
  \newtheorem{theorem}{Theorem}[section]
  \newtheorem{conjecture}[theorem]{Conjecture}
  \newtheorem{problem}[theorem]{Problem}
  \newtheorem{assumption}[theorem]{Assumption}
  \newtheorem{heuristic}[theorem]{Heuristic}
  \newtheorem{proposition}[theorem]{Proposition}
  \newtheorem{fact}[theorem]{Fact}
  \newtheorem{lemma}[theorem]{Lemma}
  \newtheorem{corollary}[theorem]{Corollary}
  \newtheorem{claim}[theorem]{Claim}
 % \newtheorem{problem} [theorem]{Question}

\theoremstyle{definition}
  \newtheorem{definition}[theorem]{Definition}
  \newtheorem{example}[theorem]{Example}
  \newtheorem{remark}[theorem]{Remark}
  \numberwithin{equation}{section}

\begin{document}
\title{Math 104 - Weekly assignment 2}
\author{Due 9 September 2016, by 16:00}
\maketitle

\begin{enumerate}
\item Show that, for any $a,b\in\R$ with $a<b$, there exist infinitely many irrational numbers $x$ with $a<x<b$.
\vspace{0.4in}
\item Show that, for any $a, b\in\R$, it holds that
$$\big| |a|-|b| \big|\leq |a+b|
$$
and 
$$\big| |a|-|b| \big|\leq |a-b|.
$$
(Hint: These are both corollaries of the triangle inequality.)
\vspace{0.4in}

\item Show the Cauchy--Schwarz inequality: For any $a_1,\ldots,a_n\in\R$, $b_1,\ldots,b_n\in\R$, 
$$(a_1b_1+\ldots+a_nb_n)^2\leq (a_1^2+\ldots +a_n^2)\cdot (b_1^2+\ldots +b_n^2).
$$
(Hint: Consider the polynomial 
$$p(\lambda)=(a_1+\lambda b_1)^2+\ldots +(a_n+\lambda b_n)^2,\; \lambda \in \R.
$$
What is the degree of $p$? What is the sign of $p(\lambda)$ for $\lambda\in\R$? What does this imply for the discriminant of $p$?
\vspace{0.4in}

\item Use the geometric-arithmetic mean inequality to show the harmonic-geometric mean inequality. I.e., use that
$$(x_1\cdots x_n)^{1/n}\leq \frac{x_1+\ldots+x_n}{n}\text{, for all }x_1,\ldots,x_n >0
$$
to prove that
$$\frac{n}{\frac{1}{x_1}+\ldots +\frac{1}{x_n}}\leq (x_1\cdots x_n)^{1/n}\text{, for all }x_1,\ldots,x_n>0.
$$
\vspace{0.4in}

\item Let $(a_n)_{n\in\N}$ be a sequence. Show that $(a_n)_{n\in\N}$ is bounded if and only if there exists some $M>0$ with $|a_n|\leq M$, for all $n\in \N$.
\vspace{0.4in}

\item Let $(a_n)_{n\in\mathbb{N}}$ be a sequence. (The aim of this exercise is to show that the convergence behaviour of a sequence doesn't depend on the first terms of the sequence.)
\begin{enumerate}[(i)]
\item Suppose that $(a_n)_{n\in\N}\rightarrow a$, for some $a\in\R$. Show that any final part $(a_m,a_{m+1},\ldots)$ of $(a_n)_{n\in\N}$ also converges to $a$.
\item Suppose that some final part $(a_m,a_{m+1},\ldots)$ of $(a_n)_{n\in\N}$ converges to some $a\in\R$. Show that also $a_n\rightarrow a$.
\item Deduce that:
\begin{enumerate}[(a)]
\item $a_n\rightarrow a$ if and only if $a_{n+3}\rightarrow a$.
\item If $a_n\leq b_n\leq c_n$ for all $n\geq n_0$ in $\N$, for some $n_0\in\N$, and if $a_n\rightarrow l$, $c_n\rightarrow l$, then also $b_n\rightarrow l$. (This means that we can use the sandwich lemma even if $b_n$ is not between $a_n$ and $c_n$ for finitely many $n$).
\end{enumerate}
\end{enumerate}
\vspace{0.4in}

\item Show that, for all $x \in \R$, there exists a sequence of irrational numbers converging to $x$.
\vspace{0.4in}

\item Show that $\nexists a\in\R$ with $(-1)^n\rightarrow a$ as $n\rightarrow +\infty$ (i.e, the sequence $(-1)^n$, for $n\in\N$, doesn't converge).

(Hint: Let $a\in\R$. How would we show that $a_n=(-1)^n$ doesn't converge to $a$? We would see what the negation of $a_n\rightarrow a$ is. The definition is: for any neighbourhood of $a$, I can find a final part of $(a_n)_{n\in\N}$ in the neighbourhood. So, the negation is: there exists a neighbourhood of $a$ that doesn't contain any final part of $(a_n)_{n\in\N}$. That's what we need to show for our sequence. Or, if you prefer it this way: we should show that $\exists \epsilon>0$ such that, for all $n_0\in\N$, there exists some $n\geq n_0$ in $\N$, with $|a_n-a|>\epsilon$. This the negation of the more formal definition of convergence.)
\vspace{0.4in}

\item True of False? Explain your answer:
\begin{enumerate}[(i)]
\item Every convergent sequence of irrational numbers converges to an irrational number.
\item Every bounded sequence converges.
\item If $a_n\rightarrow 0$ and $(b_n)_{n\in\N}$ is bounded, then $(a_nb_n)_{n\in\N}$ is bounded.
\item If $a_n\rightarrow a$ and $a>0$, then $a_n>0$ for large $n$ (i.e., there exists some $n_0\in\N$ such that: for all $n\geq n_0$, $a_n>0$).
\item If $a_n\rightarrow a$ and $a\geq 0$, then $a_n>0$ for large $n$.
\end{enumerate}
\vspace{0.4in}

\item 
\begin{enumerate}[(i)]
\item Let $a_n\rightarrow a$ and $b_n\rightarrow b$. Show that, if $a_n\leq b_n$ for all $n\in\N$, then $a\leq b$.

(Hint: For contradiction, suppose that $a> b$. Find two disjoint neighbourhoods of $a$ and $b$, and find final parts of the corresponding sequences in there. What goes wrong?)
\item Let $a_n\rightarrow a$ and $b_n\rightarrow b$ with $a_n\lneq b_n$, for all $n\in\N$. Is it necessarily true that $a\lneq b$?
\end{enumerate}
\end{enumerate}

\end{document}

